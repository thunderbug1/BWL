\chapter{Finanzierung}
\section{Zentrale Annahmen}
\begin{itemize}
	\item Förderungen/Zuschüsse sind nicht in den Berechnungen enthalten
	\item Keine Gewinnausschüttung bzw. Bonifikationen an die Unternehmensgründer
	\item Zahlungsziele (Kunden und Lieferanten): 30 Tage
\end{itemize}

\section{Finanzierungsmodell}
Die Finanzierung des Unternehmen erfolgt durch 4 Säulen. 

\noindent
\begin{tabular}{@{}>{\raggedright\arraybackslash}p{1.8cm}@{}>{\raggedright\arraybackslash}p{\textwidth - 1.8cm}}
 
	\textbf{Säule 1:} & Der Hersteller des Roboters finanziert die Anpassung des System an den jeweiligen Roboter mit der Entwicklungsgebühr. \\ 

	\textbf{Säule 2:} & Die Verwendung des Systems ist lizenziert. Für jeden Roboter, in dem das System zum Einsatz kommt, ist eine jährliche Lizenzgebühr fällig. Diese ist abhängig davon, wie viel Einsparungspotential durch das System möglich ist und welche Version des Systems verwendet wird. \\
	
	\textbf{Säule 3:} & Der Roboterhersteller finanziert eine Weiterentwicklung eines bestehenden Systems. \\
	
	\textbf{Säule 4:} & Damit der Betreiber das System verwenden müssen seine Mitarbeiter geschult werden.
\end{tabular}

\section{Entwicklungsgebühr}
\subsection{Neuentwicklung}
Die Entwicklungsgebühr ist unabhängig vom Roboter und Hersteller. Sie stützt sich darauf, dass der Roboter für die Betreiber attraktiver wird, da dieser eine jährliche Stromersparnis und dadurch resultierende Kostenersparnis mit sich bringt.\\
Die Entwicklungsgebühr beträgt $100.000,00$ \officialeuro.\\
Diese kann um bis zu $20$\% verringert werden sofern entsprechende Gegenleistungen angeboten werden.

\subsection{Weiterentwicklung}
Das grundlegende System unterliegt einer ständigen Weiterentwicklung. Neue Versionen des System müssen jedoch wieder an einen Roboter angepasst werden. Abhängig von den Versionsunterschieden beträgt die Gebühr zwischen $25.000$ und $50.000$ \officialeuro.\\
Diese kann um bis zu $20$\% verringert werden sofern entsprechende Gegenleistungen angeboten werden.

\section{Lizenzgebühren}
Die Verwendung des Systems verschafft dem Betreiber einen enormen Wettbewerbsvorteil. Damit der Betreiber das System verwenden kann ist eine jährliche Lizenzgebühr fällig. Diese beträgt für die \textit{Version 1} mindestens $500,00$ \officialeuro. Ab $5,00$\% steigt die Lizenzgebühr proportional zur Energieersparnis. Als zweiter Referenzpunkt gilt die $30,00$\%-Marke bei der die Lizenzgebühr $2.000,00$ \officialeuro beträgt, siehe Abbildung \ref{fig:lizenzgebuehr}.
\begin{figure}[h]
	\centering
	\includegraphics[width=10cm]{Lizenzgebuehr.pdf}
	\caption{Lizenzgebühr für die \textit{Version 1}}
	\label{fig:lizenzgebuehr}
\end{figure}

\paragraph*{Versionsupdate:}
Wir eine neue Version des Systems veröffentlicht, kann sich die Lizenzgebühr verändern. Die Unterstützung von mehreren Versionen des System und den dazugehörigen Lizenzgebührmodellen ist möglich.

\section{Schulung}
\subsection{Grundschulung}
Die Grundschulung umfasst die Installation und Verwendung des System mit der Robotersteuerung. Die Schulung dauert 2 Tage und die Kosten betragen $800,00$ \officialeuro pro Person.

\subsection{Intensivschulung}
Im System können Parameter verändert werden um mögliche spezielle Anforderungen abdecken zu können. Wie die Parameter verändert werden können und welche Auswirkung die Veränderungen haben ist Teil der Intensivschulung. Diese dauert 3 Tage und die Kosten betragen $1.500,00$ \officialeuro pro Person.