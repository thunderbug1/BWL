\chapter{Unternehmen}
\section{Informationen zum Unternehmen}
Die Firma \textbf{Robot Technology and Innovations GmbH (RTI)} wird im Jänner 2018 von vier Akademikern gegründet. Das Unternehmen beschäftigt sich mit der Entwicklung neuartiger Steuerungen, die speziell für industrielle Roboter eingesetzt werden. \\
Die Anteile der Firma wurden zu je 25\% auf die Geschäftsführer- und -innen aufgeteilt. 


%TODO Kooperationspartner, vor -und nachteile von kooperationspartner (ev. Kooperationspartner: F & E Wels?)


\section{Status der Unternehmensgründung}
Auf Grund der Rechtsform müssen wir unsere GmbH ins Firmenbuch eintragen lassen. Die Genehmigung der Gewerbeberechtigung muss noch durch das Magistrat Linz stattgegeben werden. 


\section{Firmensitz}
Der Firmensitz der \textbf{Robot Technology and Innovations GmbH}  befindet sich im Gewerbegebiet in Linz, Franzosenhausweg 49a. Ein Vorteil der Liegenschaft ist die exzellente Verkehrsanbindung, da sich unser Unternehmen gleich in der Nähe der Autobahnabfahrt Linz/Franzosenhausweg befindet. 

%\section{Unternehmensanalyse}
%TODO Kernaufgabe des UN, stärken und schwächen bzw chancen und risiken für UN, und wie planen damit umzugehen?
%TODO welches geschäftsmodell haben sie vorgesehen?
%TODO Aus welchen Einzeltätigkeiten setzt sich die Leistung zusammen (Wertschöpfung)

%Das Unternehmen ist eine Gesellschaft mit beschränkter Haftung 

\section{Organisationsstruktur}
%TODO Diagramm unserer Organisation

\section{Ziele}
Durch unsere realistischen Zielsetzungen haben wir ein gemeinsames Bild  im Kopf, was wir in Zukunft erreichen wollen. Zusätzlich haben wir unsere Ziele in kurz-, mittel- und langfristige Ziele eingeteilt. \\ 

\textbf{Kurzfristige Ziele}
\begin{itemize}
	\item Wir wollen einen Gesellschaftsvertrag abschließen, um eine GmbH zu gründen.  % Abschluss des Gesellschaftsvertrages
	\item Erfolgreiche Genehmigung der Gewerbeberechtigung durch das Magistrat Linz. %Genehmigung der Gewerbeberechtigung
	\item Die Formalitäten der Gründung abschließen, damit folgt der Eintrag ins Firmenbuch.  %Eintrag ins Firmenbuch
	\item Wir wollen in den ersten 2 Geschäftsjahren Marktführer in der Region werden. %Marktführerschaft in der Region
	\item %Gewinnung von Neukunden
	\item Durch Veranstaltungen, Messen,... wollen wir neue Geschäftskontakte knüpfen. \\ %Knüpfung von Geschäftskontakten
\end{itemize} 


\textbf{mittelfristige Ziele}
\begin{itemize}
	\item Wir wollen nach 3 Jahren ein Umsatzwachstum von 40\% gegenüber dem Vorjahr erwirtschaften. %Umsatzwachstum von 40\% gegenüber dem Vorjahr
	\item Das Unternehmen soll um eine eigene Produktions- und Testhalle erweitert werden. %Eigene Testhalle
	\item Nach 3 Jahren wollen wir am internationalen Markt teilnehmen. \\ %
	\item 
\end{itemize}

\textbf{langfristige ziele}
\begin{itemize}
	\item Nach 5 Jahren wollen wir zu den Marktführer in der EU gehören.  %Marktführerschaft in der EU
	\item Ebenfalls nach 5 Jahren wollen wir einen weiteren Standort eröffnen. %Ausbau des Standortes
	\item Mitarbeiterzuwachs \\%Mitarbeiterzuwachs
\end{itemize}

\section{Unterstützung und Hilfestellungen}
%TODO wer unterstützt sie (personell, ideell, finanziell)

\section{Gründungsteam}
%TODO Vielfalt, Erfafhrung & Kompetenzen
%TODO Team Erfahrung in Zusammenarbeit?, Wer übernimmt innerhalb des Teams welche Aufgaben?
\textbf{Alexander Balasch, BSc. Geschäftsführer} \\
Alexander absolviert gerade den Diplomstudiengang Automatisierungstechnik an der FH Oberösterreich, Campus Wels, welchen er voraussichtlich im Juli 2018 abschließen wird. Den darauf aufbauenden Bachelorstudiengang absolvierte er ebenfalls an der FH Oberösterreich. \\ %da kann man noch den Erfolg dazu schreiben. Berufserfahrung und Höhere Schule auch eintragen, vl. noch Vereinseigenschaften die von Vorteil sind. 


\textbf{Christine Bräuer, BSc. Geschäftsführerin} \\
Christine absolviert gerade den Diplomstudiengang Automatisierungstechnik an der FH Oberösterreich, Campus Wels, welchen sie voraussichtlich im September 2018 abschließen wird. Den Bachelorstudiengang absolvierte sie an der FH Oberösterreich, Campus Hagenberg in Medizin- und Bioinformatik. 
Neben ihrer schulischen Laufbahn ist sie seit 2010 Schriftführerin-Stv. und seit 2016 auch im Organisationsteam des Blasorchesters St. Valentin Steyr-Traktoren tätig, wo sie ihre organisatorischen Fähigkeiten und ihr Engagement ausgebaut hat. Seit 2014 arbeitet sie nebenberuflich regelmäßig bei den BMW Werken in Steyr, wo sie schon einige Erfahrung in den Bereichen Produktion und Fertigung als auch in der Montage sammeln konnte. \\

\textbf{Christopher Neuwirt, BSc. Geschäftsführer} \\
Christopher absolviert gerade den Diplomstudiengang Automatisierungstechnik an der FH Oberösterreich, Campus Wels, welchen er voraussichtlich im September 2018 abschließen wird. Den darauf aufbauenden Bachelorstudiengang absolvierte er ebenfalls an der FH Oberösterreich.\\ % Schulbildung und Berufserfahrung bitte noch selber ergänzen.

\textbf{Ing. Dominik Schönberger, BSc. Geschäftsführer} \\
Dominik absolviert gerade den Diplomstudiengang Automatisierungstechnik an der FH Oberösterreich, Campus Wels, welchen er voraussichtlich im Juli 2018 abschließen wird. Den Bacholorstudiengang Automatisierungstechnik am Campus Wels schloss er mit ausgezeichnetem Erfolg ab.  Nach dem Abschluss der HTL Neufelden, Schwerpunkt Automatisierungstechnik, war Dominik als Systems Engineer bei der TGW Mechanics GmbH in Wels tätig. Dort konnte er sich Wissen im Bereich Projektmanagment, Produktion, Konstruktion und Systemplanung aneignen. Ausbildung von neuen Mitarbeitern und Führung von kleinen Projektteams zählten ebenfalls zu seinen Tätigkeiten. 2011/2012 war Dominik als technischer Leiter in Dänemark und fungierte während dieser Zeit als stellvertretender Projektmanager vor Ort.\\

\subsection{Bisherige Zusammenarbeit}
Alexander, Christopher und Dominik haben drei Jahre gemeinsam an der FH Wels studiert und haben sich durch viele gemeinsame Projekte besser kennengelernt. Seit einem Jahr studieren wir alle vier gemeinsam an der FH Wels und fungieren bei einigen Projekten als zusammengehöriges Projektteam. Wir haben gleiche Ziele vor Augen und ziehen immer gemeinsam an einem Strang. 


\subsection{Erfahrungen und Fähigkeiten des Gründerteams}
Die nachfolgende Abbildung zeigt unsere Erfahrungen und Fähigkeiten, aber auch welche Kompetenzlücken wir durch externe Mitarbeiter noch schließen müssen. 

%TODO Tabellenbeschriftung, Legende für Kenntnisse (1 = sehr gute Kenntnisse, 2 = gute Kenntnisse, 3 = Grundkenntnisse, 4 = keine Kenntnisse)
\begin{tabular}{|>{\centering\arraybackslash}p{3cm}|>{\centering\arraybackslash}p{2.5cm}|>{\centering\arraybackslash}p{2.5cm}|>{\centering\arraybackslash}p{2.5cm}|>{\centering\arraybackslash}p{2.5cm}|}
	\hline 
	Gründerteam & {Alexander Balasch} & Christine Bräuer & Christopher Neuwirt & Dominik Schönberger \\ 
	\hline 
	\multicolumn{5}{|c|}{Hard Skills} \\ 
	\hline
	Finanzen \& Controlling 		&  & 2 &  & 3 \\ 
	\hline 
	Fremdsprachen 					&  & 2 &  & 2 \\
	\hline
	Marketing 						&  & 4 &  & 4 \\ 
	\hline 
	Software \& Hardware 			&  & 1 &  & 2 \\ 
	\hline 	
	Organisations- fähigkeit 		&  & 1 &  & 2 \\ 
	\hline	
	Personalwesen \& Entwicklung 	&  & 3 &  & 3 \\
	\hline
	Produktion 						&  & 3 &  & 2 \\
	\hline
	Projekt- management 			&  & 3 &  & 1 \\ 
	\hline 
	Verkauf 						&  & 4 &  & 2 \\ 
	\hline 
\end{tabular} 

%TODO Tabellenbeschriftung, Legende für Kenntnisse (1 = Fähigkeit, Erfahrung sehr gut ausgeprägt, 2= gut ausgeprägt, 3= grundsätzliche Erfahrung vorhanden, 4 = keine Erfahrung)
\begin{tabular}{|>{\centering\arraybackslash}p{3cm}|>{\centering\arraybackslash}p{2.5cm}|>{\centering\arraybackslash}p{2.5cm}|>{\centering\arraybackslash}p{2.5cm}|>{\centering\arraybackslash}p{2.5cm}|}
	\hline 
	Gründerteam & Alexander Balasch & Christine Bräuer & Christopher Neuwirt & Dominik Schönberger \\ 
	\hline 
	\multicolumn{5}{|c|}{Soft Skills} \\ 
	\hline 
	Anpassungs- fähigkeit 		&  & 2 &  & 2 \\ 
	\hline 
	Belastbarkeit 				&  & 1 &  & 1 \\ 
	\hline 
	Charisma 					&  & 2 &  & 2 \\ 
	\hline 
	Durchsetzungs- vermögen 	&  & 2 &  & 1 \\ 
	\hline 
	%(Selbst-) Disziplin &  &  &  &  \\ 
	%\hline 	
	Engagement  				&  & 1 &  & 1 \\  %und Einsatzbereitschaft
	\hline 
	Empathie					&  & 2 &  & 3 \\ 
	\hline 
	Flexibilität 				&  & 1 &  & 2 \\ 
	\hline 
	%Gewissenhaftigkeit &  &  &  &  \\ 
	%\hline 
	%Handlungskompetenz &  &  &  &  \\ 
	%\hline 
	%Interkulturelle Kompetenz &  &  &  &  \\ 
	%\hline 
	Kommunikations- fähigkeit 	&  & 2 &  & 2 \\ 
	\hline 
	Kritikfähigkeit 			&  & 2 &  & 1 \\ 
	\hline 
	%Kreativität &  &  &  &  \\ 
	%\hline 
	Konflikt- fähigkeit 		&  & 2 &  & 1 \\ 
	\hline 
	Kunden- orientierung 		&  & 2 &  & 2 \\ 
	\hline 
	Lebenslanges Lernen 		&  & 2 &  & 1 \\ 
	\hline 
	Menschen- kenntnis 			&  & 2 &  & 2 \\ 
	\hline 
	Motivation 					&  & 1 &  & 1 \\ 
	\hline 
	%Neugierde &  &  &  &  \\ 
	%\hline 
	%Offenheit &  &  &  &  \\ 
	%\hline 
	Präsentations- stärke 		&  & 2 &  & 2 \\ 
	\hline 
	%Serviceorientierung &  &  &  &  \\ 
	%\hline 
	%Selbstbeobachtung &  &  &  &  \\ 
	%\hline 
	%Selbstmanagement &  &  &  &  \\ 
	%\hline 
	%Souveränität &  &  &  &  \\ 
	%\hline 
	Teamfähigkeit 				&  & 2 &  & 2 \\ 
	\hline 
	Urteilsvermögen 			&  & 1 &  & 2 \\ 
	\hline 
	Verhandlungs- kompetenz		&  & 2 &  & 2 \\ 
	\hline 
	Verantwortung 				&  & 1 &  & 1 \\ 
	\hline 
	Zeitmanagement 				&  & 1 &  & 2 \\ 
	\hline 
	Zielorientierung 			&  & 1 &  & 1 \\ 
	\hline 
\end{tabular} 



\section{Mitarbeiter in der Start-Up Phase (bis Ende 2018)}
Für die Start-Up Phase werden folgende Mitarbeiter eingestellt.
\textbf{Festansgestellte Mitarbeiter}
\begin{itemize}
	\item 
\end{itemize}

\textbf{Sonstige Mitarbeiter}
\begin{itemize}
	\item   
	\item   Praktikanten
\end{itemize}