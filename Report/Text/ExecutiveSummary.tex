\chapter{Executive Summary}

Das Start-up Artificial Robot technology and Innovation (ARTI GmbH) das im Dezember 2017 gegründet wird ist eine Softwareschmiede welche welches sich mit der Entwicklung neuartigen Ansätzen zur Steuerung industrieller Roboter beschäftigt.

\section{Produkte und Dienstleistungen}

Laut IFR World Robotics 2017 sind aktuell etwa 2 Millionen Roboter industriell im Einsatz. Diese Zahl wird sich bis bis 2020 voraussichtlich auf 3 Millionen erhöhen. Ein Großteil der Betriebskosten dieser Roboter entfällt dabei auf Energiekosten was Investitionen in die Effizienz der Roboter attraktiv macht.

Unser Unternehmen bietet innovative Softwarelösungen zur Energie-optimierung der Roboterpfade an. Diese Algorithmen lassen sich dabei für unterschiedlichste Robotersteuerungen adaptieren und können daher den kompletten Markt bedienen. Die Software wird hierbei in Form von Plugins vertrieben welche vom Endkunden in die jeweiligen Softwaresuite des Roboterherstellers geladen werden und anschließend zur Pfadplanung verwendet werden können.

\section{Marketing}

Unser Produkt richtet sich an Firmen welche daran interessiert sind ihre Energiekosten zu senken indem sie die Pfade ihrer Roboter optimieren.

\section{Das Unternehmen}

Die ARTI GmbH wird von vier Akademikern gegründet welche nach einem einschlägigem Studium und mit einiger Projekterfahrung diese Idee verwirklichen wollen.

Jeder der vier Gründer erhält einen Anteil von 25\% am der ARTI GmbH, wobei zwei Personen als Geschäftsführer (kaufmännisch/technisch) fungieren werden. Die Rechtsform der GmbH wurde aus haftungs- und steuerlichen Gründen gewählt.

%\section{Kooperationen}

\section{Status der technischen Entwicklung}

Der derzeitige Status des Projektes ist eine Machbarkeitsstudie für die Umsetzbarkeit der Idee sowie für die Integration des Algorithmus in die Softwaresuite von ABB.

\section{Finanzierung}



\section{Potential}



\blindtext
\newpage
\blindtext