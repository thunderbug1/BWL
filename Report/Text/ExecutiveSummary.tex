\chapter{Executive Summary}

Das Start-up \textsf{Robot Technology and Innovation (RTI GmbH)} das im Juli 2018 gegründet wird ist eine Softwareschmiede welche sich mit der Entwicklung neuartigen Ansätzen zur Steuerung industriell eingesetzter Roboter beschäftigt.

\section{Produkte und Dienstleistungen}

Laut IFR World Robotics 2017 sind aktuell etwa 2 Millionen Roboter industriell im Einsatz. Diese Zahl wird sich bis bis 2020 voraussichtlich auf 3 Millionen erhöhen. Ein Großteil der Betriebskosten dieser Roboter entfällt dabei auf Energiekosten was Investitionen in die Effizienz der Roboter attraktiv macht.

Unser Unternehmen bietet innovative Softwarelösungen zur Energieoptimierung der Roboterpfade an. Diese Algorithmen lassen sich dabei für unterschiedlichste Robotersteuerungen adaptieren und können daher den kompletten Markt bedienen. Die Software wird hierbei in Form von Plugins vertrieben welche vom Endkunden in die jeweiligen Softwaresuite des Roboterherstellers geladen werden und anschließend zur Pfadplanung verwendet werden können.

\section{Marketing}

Die Hauptzielgruppe sind Firmen, die durch optimierung der Roboterbahnen Energie sparen wollen. Durch Entwicklung von Plugins/Libraries für gängige Softwaresuiten der großen Roboterhersteller lässt sich nahezu der gesamte Markt abdecken. Durch entwickeln einer Online-Optimierung für Flexible Fertigungszellen und Einsatzgebiete von variierenden Stückzahlen und einer Offline-Optimierung für fixe Anwendungen wie Fertigungsstraßen lassen sich unterschiedliche Zielgruppen ansprechen.

\section{Das Unternehmen}

Die \textsf{RTI GmbH} wird von vier Akademikern gegründet welche nach einem einschlägigem Studium und mit einiger Projekterfahrung diese Idee verwirklichen wollen.

Jeder der vier Gründer erhält einen Anteil von 25\% am der \textsf{RTI GmbH}, wobei zwei Personen als Geschäftsführer (kaufmännisch/technisch) fungieren werden. Die Rechtsform der GmbH wurde aus haftungs- und steuerlichen Gründen gewählt.

%\section{Kooperationen}

\section{Status der technischen Entwicklung}

Der derzeitige Status des Projektes ist eine Machbarkeitsstudie für die Umsetzbarkeit der Idee sowie für die Integration des Algorithmus in die Softwaresuite von ABB.

\section{Finanzierung}
Bis Ende des vierten Geschäftsjahres wird ein Umsatz (kumuliert) von über $5,2$Mio.\thinspace\officialeuro. $20$ Industrieroboter, dies entspricht $\sim12$\% der Robotertypen der $5$ wichtigsten Hersteller, werden in dieser Zeit für unser System vorbereitet. Zusätzlich werden knapp $1500$ Lizenzen an die Betreiber vergeben.\\
Der Finanzierungsbedarf für die \textsf{RTI GmbH} Geschäftsidee liegt bei $200.000$\officialeuro im Basis-Szenario. Die Finanzmittel werden vor allem für Personal und für die Anfangsinvestitionen benötigt. Dieser Bedarf soll mit Hilfe von Förderungen (FFG Basisprogramm), durch zinsgünstige Darlehen (UBG Gründerfond) und einem zusätzlichem Fremdkapital von den Gesellschaftern gedeckt werden. Kurzfristige hohe Liquiditätsengpässe werden durch einen Einmallkredit mit Bürgschaft durch die Gesellschafter abgedeckt.\\
Der Break-Even-Point wird im Basis-Szenario im Anfang des dritten Geschäftsjahr erreicht.


\section{Potential}
